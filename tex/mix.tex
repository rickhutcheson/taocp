\input notesmac

% Space-Savers
\def\MIX{{\tt MIX}}
\def\LOAD{{\tt LOAD}}
\def\STORE{{\tt STORE}}
\def\LESS{\tt LESS\rm}
\def\EQUAL{\tt EQUAL\rm}
\def\GREATER{\tt GREATER\rm}
\def\rA{{\tt rA}}\def\rX{{\tt rX}}\def\rJ{{\tt rJ}}\def\rAX{{\tt rAX}}
\def\rI{{\tt rI\#}}
\def\rIno#1{{\tt rI#1}}
\def\CM{{\tt C(M)}}
\def\OVT{{\tt OVRFLW}} % overflow toggle
\def\ON{{\tt ON}}
\def\OFF{{\tt OFF}}
\def\?{\hbox{\tt???}} % undefined values (overflow operations)


\title{THE \MIX\ MACHINE}

\hrule\kern2pt\hrule\kern2pt\hrule

%\index byte, word
\section The Machine

We start with some fundamental definitions needed in our description
of \MIX.
\item{$\triangleright$}
{\it Byte}: The basic addressable unit in \MIX, a byte is a block of
memory capable of holding between 64 and 100 distinct values.
All programs should work regardless of the size of a byte
(i.e. assume 64 values are available).

\item{$\triangleright$}
{\it (Computer) word}: is a collection of 5 bytes and a sign bit,
which is either $+$ or $-$.

\subsection Information Storage

%\index registers|of MIX, memory|of MIX, overflow|MIX overflow toggle
\MIX\ stores many pieces of information as it performs normal
operation. Some of this data may be stored in the
general-purpose storage provided by {\it registers} and {\it main
memory}. Other storage locations are specialized for efficient access
during some classes of operation.

\topic Registers. \MIX\ has 9 registers:

\centertable
{\bf \MIX\ Registers}
{\halign to \hsize{
#\hfil       & #\hfil             & #\hfil                    \cr
\bf Register & \bf Name           & \bf Size                  \cr
\noalign{\smallskip}
A-Register   & Accumulator        & 1 Word                    \cr
X-Register   & Extension Register & 1 Word                    \cr
I-Registers  & Index Registers    & 2 Bytes + Sign Bit        \cr
J-Register   & Jump Register      & 2 Bytes (Unsigned)        \cr}}


\topic The Virtual Register \rAX. In several of these operations,
the registers \rA and \rX are treated as a single, 10-byte
register, which we will call \rAX. This raises
the question of this register's {\it sign}. After all, it has two
distinct sign bits! The rule is simple:
\item{$\triangleright$} When {\tt STOR\/}ing into \rAX, both bits are
set to the same value.
\item{$\triangleright$}When {\tt LOAD}ing from \rAX, The sign bit
of \rA\ is used as the sign bit for the entire value.

\topic Main Memory. \MIX's main memory is comprised of 4000 words of
storage, 0--3999. This means that any memory location in \MIX\ can be
specified with 2 bytes, regardless of byte size.

\topic Specialized Storage. \MIX\ also has hardware to store the
following the following pieces of information:

\centertable{\bf Specialized Storage}
{\halign to .7\hsize{
#\hfil               & #\hfil                               \cr
\bf Location         & \bf Use                              \cr
\noalign{\smallskip}
Overflow Toggle      & Single bit; set to 1 if overflow has occurred, and \cr
                     & contains value 0 otherwise.                        \cr
Comparison Indicator & Indicates the result of a comparison operation.    \cr
                     & It will always be one of three values:             \cr
                     & \LESS, \EQUAL, or \GREATER.                        \cr}}

\section \MIX\ Operations

\subsection Operations for Loading Registers

\item{$\triangleright$} All \LOAD\ operations will {\it right-justify}
                     the number into the register. The bits to the
                     left of the value are {\it set to 0}.

\item{$\triangleright$} For the \rI\ registers, the loaded field must have
{\it at most 2 nonzero bytes, on the {\bf right} of the
field}. Otherwise, the value of the \LOAD\/ed register is \?.

%\index arithmetic|operations for \MIX, \rAX
\subsection Arithmetic

\MIX\ can perform arithmetic on its registers by {\tt LOAD\/}ing values from
memory and {\tt STOR\/}ing results into registers. All arithmetic is
assumed to be {integer-valued}, but floating point arithmetic may be
used by constructing the required operation ({\tt ADD,~DIV,} etc.) with
the $F\/$-field of the instruction set to $6$.

\topic Alignment of Arithmetic. All values are {\bf left-justified} when
{\tt STORE\/}d into registers.

\centertable{\bf Arithmetic Operations}
{\halign to \hsize{#\hfil& #\hfil&\tt #\hfil&#\hfil&#\hfil&#\hfil\cr
\tt C&\tt F&\bf Name&\bf Behavior
&\bf Time&\bf Notes\cr\trule
01 & FS & ADD  & $\rA \gets \rA+V$ ?      & 2  & \TODO\cr\trule
%---
01 & 6  & FADD & TBD (\S 4.2)             & 2  & TBD (\S 4.2)\cr\trule
%---
02 & FS & SUB  & $\rA \gets \rA-V$ ?      & 2  & \TODO\cr\trule
%---
02 & 6  & FSUB & TBD (\S 4.2)             & 2  & TBD (\S 4.2)\cr\trule
%---
03 & FS & MUL  & $\rAX\gets \rAX\times V$?& 10 &\TODO\cr\trule
%---
03 & 6  & FMUL & TBD (\S 4.2)             & 10 & TBD (\S 4.2)\cr\trule
%---
04 & FS & DIV  &$\rA\gets\pm\lfloor\rAX/V\rfloor$ & 12
                                          & If $V=0$ or $|V| > |\rAX|$,
                                            $\rAX\gets\?$ and
                                            $\OVT\gets\ON$\cr
   &    &      & $\rX\gets\pm\left|\rAX/V\right|{\rm mod}\ |V|$&&\cr\trule
%---
04 & 6  & FDIV & TBD (\S 4.2) & 12 & TBD (\S 4.2)\cr}}
%---

\centertable{\bf Miscellaneous Operations}
{\halign to \hsize{
#\hfil& #\hfil&\tt #\hfil&#\hfil&#\hfil&#\hfil\cr
\tt C\hfil&\tt F&\bf Name&\bf Behavior&\bf Time&\bf Notes\cr
00    &0    &NOP  & N/A             &0      & None         \cr
05    &0    &NUM  & \TODO            & 10   & \TODO         \cr
05    &1    &CHAR & \TODO            & 10   & \TODO         \cr
05    &2    &HLT  & \TODO            & 10   & \TODO         \cr}}

\centertable{\bf Byte-Shifting \MIX}
{\halign to \hsize{
#\hfil& #\hfil&\tt #\hfil&#\hfil&#\hfil&#\hfil\cr
\tt C\hfil&\tt F&\bf Name&\bf Behavior&\bf Time&\bf Notes\cr
06    &0    &LA  & \TODO            & 2
      & \TODO         \cr
06    &1    &SRA  & \TODO            & 2
      & \TODO         \cr
06    &2    &SLAX & \TODO            & 2
      & \TODO         \cr
06    &3    &SRAX & \TODO            & 2
      & \TODO         \cr
06    &4    &SLC  & \TODO            & 2
      & \TODO        \cr
06    &5    &SRC  & \TODO            & 2
      & \TODO\cr
07    &0--64&MOVE & \TODO            & $1+2F$
      & \TODO\cr}}
\centertable{\bf {\tt LOAD} Operations}
{\halign to \hsize{
#\hfil& #\hfil&\tt #\hfil&#\hfil&#\hfil&#\hfil\cr
\tt C\hfil&\tt F&\bf Name&\bf Behavior&\bf Time&\bf Notes\cr
08    &FS   &LDA  & $\rA\gets\CM$   &2
      &\TODO\cr
09--14&FS   &LD\# & $\rI\gets\CM$ &2
      &Bytes 1,~2, and~3 {\it must} be 0.\cr
15    &FS   &LDX  & $\rX\gets\CM$   &2
      & \TODO\cr
16    &FS   &LDAN & $\rA\gets-\CM$  &2
      &\TODO\cr
17--22&FS   &LD\#N& $\rI\gets-\CM$& 2
            & \TODO\cr
23    &FS   &LDXN & $\rX\gets-\CM$  & 2
      &\TODO\cr}}
\centertable{\bf {\tt STORE} Operations}
{\halign to \hsize{
#\hfil& #\hfil&\tt #\hfil&#\hfil&#\hfil&#\hfil\cr
\tt C\hfil&\tt F&\bf Name&\bf Behavior&\bf Time&\bf Notes\cr
24    &FS   &STA  & $\CM\gets\rA  $ & 2 & \TODO\cr
25--30&FS   &ST\# & $\CM\gets\rI$   & 2 & \TODO\cr
31    &FS   &STX  & $\CM\gets\rX$   & 2 & \TODO\cr
32    &FS   &STJ  & $\CM\gets\rA$   & 2 & \TODO\cr
33    &FS   &STZ  & $\CM\gets 0$    & 2 & \TODO\cr}}

\centertable{\bf \TODO}
{\halign to \hsize{
#\hfil& #\hfil&\tt #\hfil&#\hfil&#\hfil&#\hfil\cr
\tt C\hfil&\tt F&\bf Name&\bf Behavior&\bf Time&\bf Notes\cr
34    &0--20&JBUS & $\CM\gets 0$    & 1    & \TODO\cr
35    &0--20& IOC & $\CM\gets 0$    & $1+T$& \TODO\cr
36    &0--20& IN  & $\CM\gets 0$    & $1+T$& \TODO\cr
37    &0--20&OUT  & $\CM\gets 0$    & $1+T$& \TODO\cr
38    &0--20&JRED & $\CM\gets 0$    & 1      & \TODO\cr
39    &0    &JUMP & \TODO            & 1      & \TODO\cr
39    &1    &JSJ  & \TODO            & 1      & \TODO\cr
39    &2    &JOV  & \TODO            & 1      & \TODO\cr
39    &3    &JNOV & \TODO            & 1      & \TODO\cr
39    &4    &JL   & \TODO            & 1      & \TODO\cr
39    &5    &JE   & \TODO            & 1      & \TODO\cr
39    &6    &JG   & \TODO            & 1      & \TODO\cr
39    &7    &JGE  & \TODO            & 1      & \TODO\cr
39    &8    &JNE  & \TODO            & 1 & \TODO\cr
39    &9    &JLE  & \TODO            & 1 & \TODO\cr
40    &0    &JAN  & \TODO            & 1 & \TODO\cr
40    &1    &JAZ  & \TODO            & 1 & \TODO\cr
40    &2    &JAP  & \TODO            & 1 & \TODO\cr
40    &3    &JANN & \TODO            & 1 & \TODO\cr
40    &4    &JANZ & \TODO            & 1 & \TODO\cr
40    &5    &JANP & \TODO            & 1 & \TODO\cr
41--46&0    &J\#N & \TODO            & 1 & \TODO\cr
41--46&1    &J\#Z & \TODO            & 1 & \TODO\cr
41--46&2    &J\#P & \TODO            & 1 & \TODO\cr
41--46&3    &J\#NN& \TODO            & 1 & \TODO\cr
41--46&4    &J\#NZ& \TODO            & 1 & \TODO\cr
41--46&5    &J\#NP& \TODO            & 1 & \TODO\cr
47    &0    &JXN  & \TODO            & 1 & \TODO\cr
47    &1    &JXZ  & \TODO            & 1 & \TODO\cr
47    &2    &JXP  & \TODO            & 1 & \TODO\cr
47    &3    &JXNN & \TODO            & 1 & \TODO\cr
47    &4    &JXNZ & \TODO            & 1 & \TODO\cr
47    &5    &JXNP & \TODO            & 1 & \TODO\cr
48    &0    &INCA & \TODO            & 1 & \TODO\cr
48    &1    &DECA & \TODO            & 1 & \TODO\cr
48    &2    &ENTA & \TODO            & 1 & \TODO\cr
48    &3    &ENNA & \TODO            & 1 & \TODO\cr
49--54&0    &INC\#& \TODO            & 1 & \TODO\cr
49--54&1    &DEC\#& \TODO            & 1 & \TODO\cr
49--54&2    &ENT\#& \TODO            & 1 & \TODO\cr
49--54&3    &ENN\#& \TODO            & 1 & \TODO\cr
55    &0    &INCX & \TODO            & 1 & \TODO\cr
55    &1    &DECX & \TODO            & 1 & \TODO\cr
55    &2    &ENTX & \TODO            & 1 & \TODO\cr
55    &3    &ENNX & \TODO            & 1 & \TODO\cr
56    &FS   &CMPA & \TODO            & 1 & \TODO\cr
57--62&FS   &CMP\#& \TODO            & 1 & \TODO\cr
63    &FS   &CMPX & \TODO            & 1 & \TODO\cr}}

\section {\tt MIXAL} Assembly

The most powerful feature of the {\tt MIXAL} language is the ability
to give a {\it symbolic name} to the location of an instruction in
memory.

\subsection Rules for Assembly

\item{1.}
A symbol can be 1 - 10 letters / digits, but must contain at least 1 letter.
\itemitem{a)} Symbols may not be in the form
              {\tt DH},~{\tt DF},or~{\tt DB}, where {\tt D} is a
              digit, since these have special meaning to MIXAL.

%% 2. A number is a string of 1 - 10 digits.

%% 3. Symbols in MIXAL may be forward references.
%%    **LOCATION SYMBOLS MAY NOT BE RE-DEFINED**

%% 4. ???

%% 5. ???
%%      a. *Signed* atomic expression
%%      b. ???
%%      c. An expression, followed by a binary operation, followed by an atomic
%%         expression.
%%         * Allowed Binary Operations: `+`, `-`, `*`, `/`, `//`, and `:`
%%         * Binary operations can be followed left to right.

%% 6. An A-part (address part) is either
%%      a. Empty (denoting 0)
%%      b. an expression
%%      c. a future reference (see rule 13)
%%      d. a literal constant (see rule 12)

%% 7. An index part is either:
%%      a. empty (denoting 0)
%%      b. a comma, followed by an expression (denoting the value of that
%%         expression)

%% 8. An F-Part is either:
%%      a. empty (denoting normal F-settings, which changes based on the OpCode)
%%      b. A left parenthesis, followed by an expression, followed by a right
%%         parenthesis.

%% 9. A W-Value (denoting a full-word MIX constant) is either:
%%      a. An expression, followed by an F-part
%%      b. A W-Value, followed by comma, followed by a W-Value in form (a).

%% 10. `*` always refers to the value of the *location counter* at the
%%     point where `*` is found. The value of `*` should always be
%%     nonnegative, and it should fit into 2 bytes.

%% 11. (Processing the OP field)
%%      There are 6 Possibilities:

%%      a. `OP` is a MIX operation, and so its normal / default C and F
%%              values are defined. In this case, ADDRESS must be an A-part,
%%              followed by an index part, followed by an F-part.

%%      b. `OP` is `EQU`. (READ)

%%      c. `OP` is `ORIG`, a pseduo-op that sets the LC to the value specified
%%              in `ADDRESS`. If LOC is not blank, then the symbol in LOC should
%%              be set to the value of the LC   **before** it was
%%              updated. In this

%%              case, `ADDRESS` should be a W-value.

%%      4. `OP` is `CON`. `ADDRESS` should be a W-Value. The effect of this
%%         pseudo-op is to place a word with that W-Value into the memory
%%         location specified by the LC. The LC is incremented.

%%      5. `OP` is `LF`. The `ADDRESS` field should be a set of characters. The
%%         effect will be to assemlble the word of character codes formed by
%%         the first 5 characters of the ADDRESS field.

%%      6. `OP` is `END`. The `ADDRESS` should be a  W-Value, which
%%         specified the location of the instruction at which the program
%%         should begin.

%% 12. Literal constants: Any W-Value that is less than 10 characters
%%     long can be enclosed in `=` signs and used for future
%%     references. This pseudo-operation creates a new, randomly-named
%%     internal symbol and defines a constant with that symbol.

%% 13. Every symbol has one and only one equivalent value

%% # Expressions in `MIXAL`
%% Expressions are arithmetic combinations of:

%% * Numbers
%% * *Defined* Symbols
%% * `*`, the current value of the Location Counter.

%% Word Specifications are used to denote any value that can be held in a single
%% `MIX` word. They can be written as

%%      <DIGIT>   :=  0 | 1 | ... | 9
%%      <NUM>     := { DIGIT } * 10
%%      <CONS>    := =<NUM>=
%%      <BIN-OP>  := + | - | * | / | // | :
%%      <LETTER>  := a | ... | z | A | ... | Z
%%      <SYM>     := { <LETTER> | <NUM> }
%%      <ATOM>    := <NUM> | <DSYM> | *
%%      <S-ATOM>  := <ATOM> | +<ATOM> | -<ATOM>
%%      <EXPR>    := <S-ATOM> | <S-ATOM> <BIN-OP> <EXPR>
%%      <EXP-FLD> := <EXPR>(<EXPR>) | <EXPR>
%%      <IX-EXPR> := <EXPR>,<EXP-FLD>  | <EXPR-FLD>
%%      <A-PART>  := EMPTY | <F-REF> | <CONS> | <EXPR>
%%      <W-VAL>   := <EXP-FLD>,<W-VAL> | <EXP-FLD>

%% # Instruction Format
%% Almost all `MIXAL` instructions are in the following format:

%% |1 - 10 | 11  | 12 - 15 | 16  |           17 - 80                |
%% |:----: |:---:|:-------:|:---:|:--------------------------------:|
%% |`LOC`  | `_` | `OP`    | `_` | `ADDRESS` and (optional) Remarks.

%% # Transforming Instructions
%% `MIXAL` instructions is translated into words of memory as follows:

%% `LOC OP ADDR`

%% | Bit 0 | Bit 1 | Bit 2 | Bit 3 | Bit 4        | Bit 5  |
%% |:----: |:-----:|:-----:|:-----:|:------------:|:------:|
%% | $+/-$ | $A_1$ | $A_2$ |  $I$  | $F$          | $C$    |
%% | Address               | Index | Modification | OpCode |

%% ## Instruction Fields

%% ### Address (A-Field)
%% This field is self explanatory for most instructions. Usually, it refers to a
%% location in memory, but it can also be used in any situation where a 2-byte
%% value is necessary.

%% ### Index (I-Field)
%% The value of the `I` field must be a number between 0 and 6. If `I` = 0, then
%% the value in the A-field will be used by operations without
%% modification. If $1\le i\le 6$, the value in register $\tt rIi$ will
%% be added to the value of the A-field.

%% ### OpCode
%% The value here specifies which operation the `MIX` computer will execute.
%% OpCodes for each instruction are in the table below.

%% ### Modification (F-Field)
%% This holds a modification *of the OpCode*. How the modification affects the
%% operation depends on which operation is being executed.

%% * For *most* instructions, the F-field holds a field specification, in
%%   the form $(L:R) = 8L + R$. This case is common for operations that
%%   could load only *part* of a word into a register, etc.
%% * In some cases, the F-field value changes *which* operation is
%%   invoked. For example, consider opcode `48`.
%%        * If $\tt C=48$ and $F=0$, this is the operation `INCA`. It
%%         *adds* the value `C(M)` to `rA`.
%%        * If $\tt C=48$ and $F=1$, this is the operation `DECA`. It
%%         *subtracts* the value `C(M)` from `rA`.

%% # Pseudo-Operations

%% | Mnemonic | Required Format | Operation | LC Change |
%% |:-------- |:--------------- |:--------- |:--------- |
%% EQU  | [LBL] EQU WRD        | Set Equal                         | None
%% ORIG | [LBL] ORIG WRD       | Set Location Counter              | `LC = WRD`
%% CON  | [LBL] CON WRD        |  Create Constant                  | `LC += 1`
%% ALF  | [LBL] ALF [C[C[C[... | Convert Characters                | `LC += 1`
%% END  | [LBL] END WRD        | End Program; Set Initial Location | None


%% `EQU`
%% :   Sets the Symbol in the `LOC` field to the value specified in the
%%      `ADDRESS` field.

%% `ORIG`
%% :   Sets the location counter to the value of the `ADDRESS` field.

%% `CON`
%% :   Places the value specified in the `ADRESS` field into memory at
%% the location

%%      currently pointed to by the location counter.

%% `ALF`
%% :   Converts the first 5 characters of the ADDRESS field into their
%%      corresponding character codes, then stores them in memory at the address
%%      pointed to by the location counter.

%% `END`
%% :   The value of the `ADDRESS` field is used as the location of the
%% instruction

%%      at which the program begins.

\bye
