\input notesmac
\input mathmac

\font\sectionbf=cmss10 scaled \magstep1
\font\bigbf=cmssq8 scaled \magstep3
\font\bigit=cmssi12 at 18pt
\font\titlerm=cmss10   scaled \magstep3
\font\titlebf=cmssbx10 scaled \magstep3
\font\titleit=cmssi10  scaled \magstep3
\font\bodyrm=cmr12 at 14pt
\font\bodyit=cmti12 at 14pt
\font\bodybf=cmbx12 at 13pt
\let\rm\bodyrm
\let\it\bodyit
\let\bf\bodybf

\fourteenpoint

\title{1.1 Algorithms}

\hrule

\section Introduction

This entire section deals with the formal definition of algorithms,
and a brief discussion of formal correctness proofs.
Because this formal definition of an {\it algorithm} is fairly dense,
we work into it slowly, beginning with an example.

\beginalgo 1.1E (Euclid's Algorithm)
Given $m \in \Z, n \in \Z$, find their {\it greatest common divisor}.


\algostep E1. (Find Remainder) Divide $m$ by $n$, set $r$ to the remainder.
\invariant{$0 \leq r < n$}
\algostep E2. (Halt Condition) If $r = 0$, stop; $n$ is the answer.
\algostep E3. (Reduction) $m \gets n, n \gets r$; goto {\bf E1}.

\endalgo

\topic A correctness proof.

Correctness proofs, especially those involving recursion, will, in
many cases, use inductive reasoning.

After step {\bf E1}, we've calculated $q \in \Z^+$,
$r \in \Z^{nonneg}\ $ satisfying $m = qn + r$. \TODO


\section Computational methods \& their properties

It is useful to define how ``useful'' or ``correct'' a ``recipe'' is,
but for that we need metrics by which to measure it. We define some
useful metrics below, and then define some categories for our
``computer recipes'' based on these properties.

\topic Some Useful Properties.

\item 1 {\bf Finiteness} -- When strictly defined, algorithms must be
guaranteed to terminate (in finite time / steps). To prove this, we
usually prove that the quantity / value used in the {\it halting
condition} changes in some way. Usually, we prove that it decreases.

\item 2 {\bf Definiteness} -- Every step must be explicitly defined
(any programming language is sufficient, unless there is undefined
behavior involved).

\item 3 {\bf Inputs} -- Zero or more inputs, take from explicitly specified {\it sets}. Algorithm E's inputs come from the set $\Z$.

\item 4 {\bf Outputs} -- One or more outputs, having some relation to the
inputs. Specifically, the ``relation'' is the algorithm itself.

\item 5 {\bf Effectiveness} -- This is a general measure of
``simplicity''. All operations within the method must be basic enough
to be performed with pencil \& paper, {\bf or} must be computed with a
known method that is also {\it effective}. \aside{This is one of the biggest
differences between proofs in mathematics and computational methods.}

\topic Categories.

\definition(Computational method) A procedure with to all of the
above properties except for {\bf finiteness}. These can be useful in
their own right; for example, {\it reactive processes}, which
continually react to inputs or changes in their environment, do not
have a finite endpoint, and that's exactly the way we want it!

\definition(Algorithm) A type of computational method that exhibits
all 5 of the above properties.

\definition(Program) A computational method expressed in a computer
language.

\section The Formal Definition

We'll first define general computational methods, then restrict that
definition to fit only algorithms.

\definition(Computational Method)
A 4-tuple, $(Q, I, \Omega, f)$ whose components are defined as follows:

\medskip\item{$I$} is the set of {\it input states}
\medskip\item{$\Omega$} is the set of {\it output states}
\medskip\item{$f$} is the {\it computational function}; it maps $Q \to Q$,
               and is {\it pointwise fixed} in $\Omega$. (That
               is, $\forall q \in \Omega, f(q) = q$).
\medskip\item{$Q$} is the set of {\it computational states}. These are the
               ``internal states'' of our procedure, both explicit and
               implicit. This is tricky, and it's important to
               remember that even the ``current step'' we're on in the
               process is part of that state.

\topic Computational states \& sequences.

$Q$ is really the big player of this definition, because it
encompasses {\bf all possible states} the ``algorithm machine''
can be in. Note that this implies both $I \subseteq Q$ and
$\Omega \subseteq Q$.

Let's say we have an input, $x \in I$, and we want to follow the
method along to the desired output, $y \in \Omega$. We'll apply $f$ to
x, see what state this generates, and apply $f$ again. We'll continue
to do this until we reach $y$, creating a {\it computational
sequence}:
$$(x_0, x_1, x_2, \ldots, x_n) \quad \hbox{s.t.}\quad x_{k+1} = f(x_{k})$$

Now, we can formally say that an {\bf algorithm} is a computational
method whose computational sequences always end in a finite $k$
steps.

\section Expressing Euclid's Algorithm Formally

\topic Defining $I$.

Let's start with the simplest translation, inputs and outputs.

We already know that our inputs are always 2 integers, $m$ and
$n$. We'll translate that to an ordered pair, $(m, n)$.

$$I = (m, n) \quad m\in\Z, n\in\Z$$

\topic Defining $\Omega$.

This is even simpler. We'll output $gcd(m, n)$, a single nonnegative
integer.

$$\Omega = (n) \quad n\in\Z^{nonneg}$$

\topic Defining $Q$ \& $f$.

Now, we know that $Q$ has to contain both $I$ and $\Omega$. It {\it
also} needs to contain the intermediate states of the calculation. We
don't have much state in the algorithm; mostly just $r\in\Z^{nonneg}$.
Don't forget the implicit state of ``current step'' of the algorithm.

We'll encode the state into 4-tuples of the form $(m, n, r, c)$. $m$,
$n$, and $r$ have the same meaning as the algorithm. $c$ encodes the
``next line'' of the algorithm.

$$\eqalign
{
f((m, n)) &= (m, n, 0, 1)\cr
f((m, n, r, 1)) &= (m, n, \hbox{ remainder of } m/n, 2)\cr
f((m, n, r, 2) &= \cases{(n) &if r = 0\cr
                        (m, n, r, 3) &if r \neq 0}\cr
f((m, n, p, 3)) &= (n, p, 0, 1)
}
$$
\bigaside{Knuth uses {\it p} for the new {\it r} value instead of {\it
0}. However, that's unnecessary, since the {\it c = 1} case does not
rely on the value of {\it r}.
}
\bye
