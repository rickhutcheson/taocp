%%
%% SELECTED ANSWERS TO VOLUME 1 OF THE ART OF COMPUTER PROGRAMMING
%%
%% Includes (at least sketches of) all of the answers that I have
%% worked in TAOCP.
%%
\input notesmac % Use notes format, with additional commands below.
\font\sectionbf=cmss10 scaled \magstep1
\font\titlerm=cmss10   scaled \magstep3
\font\titlebf=cmssbx10 scaled \magstep3
\font\titleit=cmssi10  scaled \magstep3

%%
%% Structure & Formatting
%%

\def\answersection #1{\bigskip\noindent{\sectionbf SECTION #1}
\smallskip\hrule height2pt\smallskip}
\def\question #1 -- {\bigskip\noindent{\bf #1 \quad}}
\def\aside(#1){({\it#1})}

%%
%% MIX Shortcuts & Operations
%%

\def\MIX{{\tt MIX}}
\def\LOAD{{\tt LOAD}}
\def\STORE{{\tt STORE}}
\def\LESS{\tt LESS\rm}
\def\EQUAL{\tt EQUAL\rm}
\def\GREATER{\tt GREATER\rm}
\def\rA{{\tt rA}}\def\rX{{\tt rX}}\def\rJ{{\tt rJ}}\def\rAX{{\tt rAX}}
\def\rI#1{{\tt rI\#}}
\def\rIno#1{{\tt rI#1}}
\def\CM{{\tt C(M)}}
\def\OVT{{\tt OVRFLW}} % overflow toggle
\def\ON{{\tt ON}}
\def\OFF{{\tt OFF}}
\def\?{\hbox{\tt???}} % undefined values (overflow operations)


%%
%% OPENING CONTENT
%%

\title {\bf SELECTED ANSWERS}

\par\title \it The Art of Computer Programming, Vol. I

%%
%% 1.1
%%

\answersection{1.1}
\question 1 -- This question demonstrates a general procedure to cycle
variables using a single temporary variable.

$$t \gets a,\quad a \gets b,\quad b \gets c,\quad c \gets d,\quad d \gets t$$

\question 2 -- This is relatively simple. If $n > m$,  then $m/n = 0
\hbox{ rem. } m$. Then, the reduction step {\bf E3} swaps $m$ and $n$,
so we are left with the "swapped" problem.

\question 3 -- Algorithm E uses the "reduction" assignment so that it
can reuse steps {\bf E1} and {\bf E2}. If we remove that assignment,
we must add redundant steps.


\beginalgo 1.1F (No-Assignment Euclid's Algorithm)

\algostep F1. (Find Remainder) Divide $m$ by $n$, set $m$ to the remainder.
\algostep F2. (Halt Condition) If $m = 0$, stop; $m$ is the answer.
\algostep F3. (Find Remainder 2) Divide $n$ by $m$, set $n$ to the remainder.
\algostep F4. (Halt Condition 2) If $n = 0$, stop; $n$ is the
answer. Otherwise goto {\bf F1}.

\endalgo

\question 4 -- The answer is $57$. The steps are:
$$
(6099, 2166) \to
(2166, 1767) \to
(1767, 399) \to
(399, 171) \to
(171, 57) \to
57
$$

\question 5 -- The procedure (at least) fails to satisfy:

\itemitem{*} finiteness - The procedure is an infinite loop
\itemitem{*} definiteness - Steps are too vague
\itemitem{*} inputs / outputs - No explicit inputs or outputs

Also, depending on some of the exercises, the steps may not actually
be effective either!

\question 6 -- As shown in the text, it suffices to average

$$T_n \quad : \quad n \le 5$$

This reduces to $$ {T_1 + T_2 + T_3 + T_4 + T_5 \over 5} = {2 + 3 + 4
+ 3 + 1 \over 5} = {13 \over 5} = 2.6$$

\question 7 -- $U_m$ \TODO

%%
%% 1.3.1
%%

\answersection{1.3.1}

\question 1 -- Since \MIX's specification requires a single ``byte''
to store no fewer than 64 value and no more than 100, we must choose a
value of $4$ ``trits'' in our trinary computer, since
$$3^3=27 < 64 < 3^4=81 < 100 < 3^5=243$$

\question 2 -- As stated in the last answer, we know that the vaguaries
of \MIX's bytes require that a single byte may only hold $64$
values. From Page 125, we see that 4 adjacent bytes hold $16,777,215$
values (not enough) and 5 of them hold $1,073,741,823$. Thus, 5 bytes
is always enough.

\question 3 -- The only useful piece of information we gain from this
question is Knuth's definition of {\it Address Field}, which we can
now assume is $\pm A$, not simply $A$.
\item{a.}$\pm A \longrightarrow (0:2)$
\item{b.}$I \longrightarrow (3:3)$
\item{c.}$F \longrightarrow (4:4)$
\item{d.}$C \longrightarrow (5:5)$

\question 4 -- The instruction requires that the memory location
specified by {C(M)} is positive, and we know that the {\tt C(M)} value
is always {\it indexed} before being used. Thus, so long as
${\tt C(I4)}>0$, this is a valid instruction.

\question 5 -- The decoding process is simple; {\tt DIV -80,3}

\question 6 --
\item{a)} $\rA \gets${\tt |-|5|1|200|15|}
\item{b)} $\rIno2\gets${\tt |-|200|}
\item{c)} $\rA\gets${\tt |+|0|0|5|1|?|}\smallskip
\smallskip
\aside(Since we don't know whether \MIX\ is a decimal or binary computer,
we can't assume how the ``200'' is stored in the word.)
\item{d)} {\bf Undefined}, because index registers can only hold 2 bytes and
a sign bit. The result of storing this many non-zero bytes is
undefined.
\item{e)} {\tt |-|0|0|0|0|0|}

\question 7 -- (Almost) directly from the text:
               $\tt rA\gets\pm\lfloor rAX/V \rfloor$;
               $\tt rX\gets\pm\left|\tt rAX/V\right|{\rm mod}|V|$
               \smallskip
               \aside(For more information, see my \MIX\ summary.)
\question 8 -- Operating on packed fields is tricky. We must work

\bye
